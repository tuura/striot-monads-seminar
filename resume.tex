\input{preamble.tex}

\usepackage[cache=false]{minted}

\title[]{Structuring Effectful Computations}

\subtitle{}

% \author[Георгий Лукьянов]{%
% Георгий Лукьянов\texorpdfstring{\\}{ }
% georgiy.lukjanov@gmail.com}

% \author[]{%
% Георгий Лукьянов\texorpdfstring{\\}{ }
% \textit{georgiy.lukjanov@gmail.com}\texorpdfstring{\\}{ }
% Артём Пеленицын \texorpdfstring{\\}{ }
% \textit{apel@sfedu.ru}
% }


\author[Georgy\,Lukyanov]
{%
  \texorpdfstring{
      % \centering
      Georgy Lukyanov\\ \scriptsize{\textit{g.lukyanov2@ncl.ac.uk}}
  }
  {Georgy Lukyanov}
}

\date{1 November 2017}%

\institute[]{%
Newcastle University \texorpdfstring{\\}{ }
Tuura\texorpdfstring{\\}{ }
}

% рисунок на титуле
% \titlegraphic{%
% % \vspace{-1.3cm}\imgw{4cm}{stannes.jpg} \hfill %
% %\vspace{-1.3cm}\imgw{3.6cm}{sheldonian.jpg}%
% }

\begin{document}

\begin{frame}
\titlepage
\end{frame}

%%%%%%%%%%%%%%%%%%%%%%%%%%%%%%%%%%%%%%%%%%%%%%%%%%%%%%%%%%%%%%%%%%%%%%%%%%%%%%%%
%%%%%%%%%%%%%%%%%  Intro  %%%%%%%%%%%%%%%%%%%%%%%%%%%%%%%%%%%%%%%%%%%%%%%%%%%%%%
%%%%%%%%%%%%%%%%%%%%%%%%%%%%%%%%%%%%%%%%%%%%%%%%%%%%%%%%%%%%%%%%%%%%%%%%%%%%%%%%

\section{Why control side effects?}

\begin{frame}[fragile]{What is a side effect}
  \begin{block}{Effectful function in an impure language}
  \begin{minted}{csharp}
int plus(int x, int y) {
  printf("Unstructured side effect");
  return x + y;
}
  \end{minted}
  \end{block}
  \begin{block}{Pure Haskell function}
  \begin{minted}{haskell}
plus :: Int -> Int -> Int
plut x y = x + y
  \end{minted}
  \end{block}
  \begin{block}{Effectful Haskell function}
  \begin{minted}{haskell}
plusIO :: Int -> Int -> IO Int
plutIO x y = do
  print "Structured side effect."
  return (x + y)
  \end{minted}
  \end{block}
\end{frame}

\section{Haskell and effects}

\begin{frame}[fragile]{Functor and Monad}
  \begin{block}{Effectful function in an impure language}
  \begin{minted}{csharp}
class  Functor f  where
    fmap :: (a -> b) -> f a -> f b
  \end{minted}
  \end{block}
  \begin{block}{Pure Haskell function}
  \begin{minted}{haskell}
class Functor m => Monad m where
    (>>=)  :: m a -> (a -> m b) -> m b
    (>>)   :: m a -> m b -> m b
    return :: a -> m a
  \end{minted}
  \end{block}
\end{frame}

\begin{frame}[fragile]{do-notation}
  \begin{block}{}
  \begin{minted}{haskell}
main :: IO ()
main = do
  putStrLn "What is your name?"
  putStrLn "|> "
  name <- readLine
  putStrLn $ "Hello, " ++ name ++ "!"
  \end{minted}
  \end{block}
\end{frame}

\begin{frame}[fragile]{MonadReader: static environment}
  \begin{block}{MonadReader type class}
  \begin{minted}{haskell}
class Monad m => MonadReader r m | m -> r where
    -- | Retrieves the monad environment.
    ask   :: m r

    -- | Executes a computation in a modified environment.
    local :: (r -> r) -> m a -> m a

    -- | Retrieves a function of the current environment.
    reader :: (r -> a) -> m a
  \end{minted}
  \end{block}
\end{frame}

\begin{frame}[fragile]{MonadRandom: random values}
  \begin{block}{MonadRandom type class}
  \begin{minted}{haskell}
class (Monad m) => MonadRandom m where
  getRandomR :: (Random a) => (a, a) -> m a

  getRandom :: (Random a) => m a

  getRandomRs :: (Random a) => (a, a) -> m [a]

  getRandoms :: (Random a) => m [a]
  \end{minted}
  \end{block}
\end{frame}

\section{Programming with controlled effects}



\begin{frame}[fragile]{Zombie apocalypses}
  \begin{figure}
    \centering
    \def\svgwidth{\columnwidth}
    \input{people_and_zombey.pdf_tex}
\end{figure}
\end{frame}

\begin{frame}[fragile]{Problem: randomly split the group in pairs}
  \begin{figure}
    \centering
    \def\svgwidth{\columnwidth}
    \input{people.pdf_tex}
\end{figure}
\end{frame}

\begin{frame}[fragile]{Domain types}
  \begin{block}{}
  \begin{minted}{haskell}
type Survivor = Int

type Group = [Survivor]

-- | Survivors keep watch in pairs
type Shift = (Survivor, Survivor)

-- | One night's schedule has three shifts
type Schedule = (Shift, Shift, Shift)
  \end{minted}
  \end{block}
\end{frame}

\begin{frame}[fragile]{Shuffling shifts}
  \begin{block}{}
  \begin{minted}{haskell}
-- | An infinite stream of random possible shifts
shifts :: (MonadReader Group m, MonadRandom m) =>
          m [Shift]
shifts = concat . repeat <$> step
  where
    step = do
      people <- ask
      possibleShifts <- shuffleM $ pairs people
      pure possibleShifts

      pairs :: [a] -> [(a, a)]
      pairs l = [(x,y) | (x:ys) <- tails l, y <- ys]
  \end{minted}
  \end{block}
\end{frame}

\begin{frame}[fragile]{Building the schedule}
  \begin{block}{}
  \begin{minted}{haskell}
-- | Construct a randomised schedule
buildSchedule :: (MonadReader Group m, MonadRandom m) =>
  m Schedule
buildSchedule = do
    [x, y, z] <- take 3 <$> shifts
    pure $ (x, y, z)
  \end{minted}
  \end{block}
\end{frame}

\begin{frame}[fragile]{Handling effects}
  \begin{block}{}
  \begin{minted}{haskell}
main :: IO ()
main = do
  let people = [1..5]
  schedule <- evalRandTIO $
    runReaderT buildSchedule people
  putStrLn $ "Watch schedule for this night: " ++
             show schedule
  \end{minted}
  \end{block}
\end{frame}

\section{Programming WITHOUT controlled effects}

\begin{frame}[fragile]{Domain types}
  \begin{block}{}
  \begin{minted}{cpp}
using Survivor = int;

using Group = vector<Survivor>;

using Shift = tuple<int, int>;

using Schedule = tuple<Shift, Shift, Shift>;
  \end{minted}
  \end{block}
\end{frame}

\begin{frame}[fragile]{Silently shuffling shifts}
  \begin{block}{}
  \begin{minted}{cpp}
vector<Shift> shifts (const Group & people) {
    auto ps = pairs(people);
    random_shuffle(ps.begin(), ps.end(), randomGen);
    return ps;
}
  \end{minted}
  \end{block}
\end{frame}

\begin{frame}[fragile]{Explicately treading the parameter}
  \begin{block}{}
  \begin{minted}{cpp}
Schedule buildSchedule (const Group & people) {
    auto shs = shifts(people);
    return make_tuple(shs[0], shs[1], shs[2]);
}
  \end{minted}
  \end{block}
\end{frame}

\begin{frame}[fragile]{Summary}
  \begin{itemize}
    \item Track effects in types.
    \item Implicitly execute service code.
    \item Use \mintinline{haskell}{do}-notation to build "imperative"~DSLs.
  \end{itemize}
\end{frame}

\begin{frame}{References}

\begin{itemize}
  \item Monadic Parser Combinators // \textit{Graham Hutton}, \textit{Erik Meijer} – Department of Computer Science, University of Nottingham, 1996
  \item These slides and code \url{https://github.com/geo2a/striot-monads}
\end{itemize}

\end{frame}

\end{document}
